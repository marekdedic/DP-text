\chapter{Evaluation on a corporate dataset}\label{chap:cisco-dataset}

The tree methods were finally evaluated on a corporate dataset in the domain of computer security.

\section{The used dataset}

The models were evaluated on a proprietary dataset provided by Cisco Cognitive Intelligence, consisting of records of network connections from clients (e.g. user computers or mobile devices) to some on-line services. The dataset represents HTTP traffic of more than 100 companies. Two datasets were collected, each spanning 1 day of traffic. The training data was traffic from 2019-11-18, while the data used for testing was collected the following day, 2019-11-19.

The dataset contained the following information for each connection:
\begin{enumerate}
  \item The duration of the connection, that is the difference between the time of the first and the last recorded datagram
  \item The client port number
  \item The IANA assigned internet protocol number
  \item The server port number
  \item The number of bytes sent from the client to the server
  \item The number of bytes sent from the server to the client
  \item The number of packets sent from the client to the server
  \item The number of packets sent from the server to the client
  \item The number of individual TCP connections
  \item The number of SYN packets sent from the client to the server
  \item The number of SYN-ACK packets sent from the server to the client
  \item The number of RST packets sent from the client to the server
  \item The number of RST packets sent from the server to the client
  \item The number of FIN packets sent from the client to the server
  \item The number of FIN packets sent from the server to the client
\end{enumerate}

In addition to that, each connection contains the URL the client is connecting to. \cite{pevny_nested_2020}
