\chapter*{Conclusion}
\addcontentsline{toc}{chapter}{Conclusion}

In this work, an overview of multi-instance learning was provided with a mathematical formalism for describing it and a survey of the paradigms and methods. Clustering was introduced as a key task in unsupervised learning. Then, three methods for clustering based on multi-instance representations were introduced, one unsupervised and two supervised. For each of the methods, the prior art it builds on was presented, along with its modification for the purposes of multi-instance clustering. All three of the methods were theoretically and experimentally evaluated and compared. The experiments were conducted first on publicly available datasets in a reproducible fashion. Following that, a corporate dataset of network security data was used as it is the primary application in mind for this work.

Comparing the methods on publicly available datasets shows the method base on triplet loss to be the best performing one. This method outperformed the other two in the sense that it was the only one to improve its performance over the learning period. This is a surprising result, as magnet loss, which builds on triplet loss and enhances it, performed worse. When evaluating on the corporate dataset however, the method based on magnet loss outperformed both of the other methods, albeit none of the results were as good as anticipated. As this problem was a much harder one, this might hint at triplet loss not being able to handle such complex tasks, while the performance of magnet loss isn't impacted as much, making it surpass triplet loss. The method based on contrastive predictive coding performed poorly on both the publicly available datasets and the corporate one. However, the comaprison might not be fair as the CPC method is unsupervised, whereas the other two can utilize labels on the training data.

Clearly, more research is needed. The most promising method, CPC, performed poorly and should be investigated more. A thorough theoretical investigation of the method including finding its local extrema might explain its low performance, however, it was judged to be outside the scope of this work. Combining the representative power of multi-instance learning with the versatility of the clustering algorithms remains an open problem.
